% Chapter 1

\chapter{Introduction} % Main chapter title

\label{Chapter1} % For referencing the chapter elsewhere, use \ref{Chapter1}

%----------------------------------------------------------------------------------------
%	SECTION 1
%----------------------------------------------------------------------------------------

In order to properly introduce this project we need to briefly cover multiple subjects, ranging from stroke rehabilitation to motion capture techniques. All these seemingly different subjects revolve around the idea that we can do something to help stroke patients recover faster by having them participate to serious Virtual Reality (VR) games in which their movements are distorted to keep them motivated.

\section{Stroke Rehabilitation}
%Explanation of what a stroke is and how to recover.
As explained by \cite{nhlbi2017what}, a stroke is a medical condition occurring when parts of the brain stop being provided with proper blood flow. Without such oxygenation, brain cells quickly die. The resulting brain damage may induce various symptoms, such as loss of vision to one side or paralysis. It is the latter that is of interest to us, and more precisely the motor recovery process involved after the stroke itself has been identified and treated. As exposed by \cite{vos2015global}, there were more than 10 million stroke cases in 2013, which represents an increase of around 60-75\% with respect to 1990.

The motor recovery process involves so-called constraint-induced movement therapy, a technique that is at least 100 years old and was proposed by \cite{oden1918systematic}. It takes advantage of neuroplasticity and involves movement exercises of the paralyzed limb. The more the movements are exercised the better the motor recovery will be.

The process as a whole is a long one, involving checkups and exercises at both the hospital and home. Many factors play a role in how successful the recovery process will be, one of them and maybe the most important one being motivation. As argued by \cite{flores2008improving}, the more a patient participates to a rehabilitation task the greater the motor recovery will be. Keeping participants motivated during such tasks is thus essential.

\section{Virtual Reality}

We now jump to a completely different topic, but a careful reader will quickly understand how both are closely linked and of interest to us.

Virtual Reality can be defined as ``The computer-generated simulation of a three-dimensional image or environment that can be interacted with in a seemingly real or physical way by a person using special electronic equipment, such as a helmet with a screen inside or gloves fitted with sensors.'' \cite{oxford2015} The most common type of VR device used nowadays are Head-Mounted Displays (HMD), wich are constructed as a screen in front of which two lenses are fixated, allowing the device to be held in front of the eyes while focusing the screen's content at infinity. Coupled with inertial sensors, such device allows one to look around at a virtual environment. Other VR displays also exist, such as the CAVE: a cube with screen-faces surrounding a user proposed by \cite{cruz1992cave}. In the recent days, companies such as Oculus VR and HTC have begun commercializing HMD and VR becomes more widespread than ever.

\subsection{Immersion, Embodiment, and Presence}
As VR becomes more and more accessible to the general public and broadly used in the industry, immersion tends to be generally accepted as a term describing all of the following concepts.

\subsubsection{Immersion}
Immersion however has a clear definition offered among others by \cite{slater2003note,sanchez2005presence} that we think is preferable to be recalled here: it refers to the capability of a system to deliver a convincing set of sensory stimuli. It is an objective measurement of parameters such as screen resolution, audio equipment, and sensors used.

\subsubsection{Embodiment}
\label{sec:embodiment}
Embodiment is defined in the field of cognitive neuroscience and philosophy of the mind by \cite{blanke2009full,debarba2017embodiment}, and encompasses the relevance of sensorimotor skills and the role the body has in shaping the mind, as well as the subjective experience of using and `having' a body. It is formally defined by \cite{de2011embodiment} as follows: ``E is embodied if some properties of E are processed in the same way as the properties of one’s body''.

One may from that perspective embody a tool, such as a pen or a hammer, even though that tool is not considered as being part of one's body. As described by \cite{de2011embodiment}, the \textit{sense} of embodiment (SoE) refers to the fact that one \textit{feels} such phenomena as opposed to only \textit{knowing} that it exists. As an example, learning that an organ is part of our body makes us embody that organ even though we cannot feel it, whereas felling like the tip of our pen actually is part of our hand creates a SoE towards that pen.

\subsubsection{Presence}
What most people tend to call immersion actually is the sense of presence, described by \cite{held1992telepresence,slater1993representations} as the "sense of being there". As explained by \cite{debarba2017embodiment}, the central concept of the state of presence is that despite \textit{knowing} that it is a simulation, the user \textit{acts in} and \textit{reacts to} the virtual environment as if it were real.

\subsection{Haptics and Self-haptics}

`Haptic' relates to the sense of touch, and more specifically to the one associated to manipulating objects. Feeling the pressure of a sphere between one's fingers when grasping a ball is a haptic feedback. Self-haptic similarly denotes the haptic feedback of touching one's own body part. The feedback thus is dual: by touching one's own arm the brain will receive both the information of the hand touching something and the arm being touched by something. Moreover, before the contact even occurs the brain predicts such self-contact using proprioceptive information about the position of both body parts in space, and is thus very sensitive to such feedback inconsistency.

\section{Motion Capture}
\label{sec:mocap}
Motion capture can be defined as the action of capturing one's movements in order to reproduce such a motion in one context or another. Video game animations for instance are often performed by actors, whose walking or fighting movements are captured and then played back on the game's characters. The captured movement is often slightly altered, e.g.\ in order to fit it onto an avatar of different morphology, as proposed by \cite{molla2017egocentric}.

\section{Amplified Embodiment}

The concept of amplified embodiment can easily be understood as the merging of both section \ref{sec:embodiment} and \ref{sec:mocap}. By capturing one's movements and transposing those on an avatar, we may create a SoE towards that avatar. The goal of amplified embodiment is to distort that avatar's movements by amplifying them while preserving the SoE, which might be partially or completely lost if the distortion is too heavy or non-continuous for instance.

\section{Serious Games}

To conclude this chapter we take a look at another distinct topic, one that closes the loop and creates a connection between stroke rehabilitation, VR, motion capture, and amplified embodiment.

As defined by \cite{djaouti2011classifying,chen2005proof}, a serious game is a game that neither has entertainment, enjoyment, nor fun as its primary purpose. Instances of serious games include educational ones such as \cite{epsitec1988blupi}, a 1988 video game where young children could learn the alphabet by exploring a house with a yellow, egg-shaped character and playing various mini games, or the more recent \cite{six2017zombie}, which is a running mobile application using storytelling coupled with GPS localization to keep people motivated at running outside.

We hope that by now the thread linking all of the above subjects is clear enough: we are interested in discovering by how much we can distort one's movements in the context of a VR serious game aimed at motor recovery while altering the sense of embodiment as little as possible, with the goal of keeping the patients motivated at performing the rehabilitation task.
