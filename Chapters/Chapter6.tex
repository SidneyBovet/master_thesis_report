% Chapter Template

\chapter{Conclusion} % Main chapter title

\label{Chapter6}

We believe the content of this report is promising. The results obtained in terms of distortion behavior are satisfying and preliminary testing was very interesting and corroborates the findings of Debarba \cite{debarba2017embodiment}, stating that a helping distortion is more easily accepted than a hindering one.

While additional work is required before a complete understanding of this phenomenon occurs, we are confident that the second part of this project will lead to interesting findings and will eventually start a new ear in stroke rehabilitation therapy.

We could for instance imagine a simplified calibration process involving only the upper limbs, along with a cheaper tracking solution such as the ones proposed by consumer-grade VR products. That would allow patients to train various movements at home and thus would dramatically reduce the cost of such therapy while allowing the clinicians to have remote access to detailed reports on their patients, with metrics like movement speed or distance covered with the affected limb. Such metrics may also be key in keeping the patients motivated, in that they can track the said measurements and see how they are improving.

\section{Further Work}

Before being able to issue a rehabilitation VR application, we must better understand to what extent the human perception allows such distortions to happen. This is going to be achieved through performing the experiment described in Chapter \ref{Chapter4} as a pilot study, and then adapting the protocol to the results we will find. Points that may need refinement include the method and metric used (adaptive staircase and gain), the different target placements, and other experiment details such as the question asked.

A prerequisite to running the experiment however is to solve the issues exposed in section \ref{sec:issues}. Priority will be given to fixing these.

In parallel, a student's project will be focusing on the calibration process, which we hope to make faster and more accurate. If the project is successful, it will be included in our procedure.

On a wider scope, additional investigation will be required in order to fully understand the impact of this and other distortions on the Sense of Embodiment, and more precisely on its Agency component. More complex interactive experimentation should be conducted, and it might be interesting to assess the SoE trough physiological measurements (e.g.\ galvanic skin response) while manipulating the strength of the distortion. Such measurement have already been carried out by Debarba et al. \cite{debarba2017perspective} in the context of perspective changes, and were found useful in assessing the differences between the conditions.
