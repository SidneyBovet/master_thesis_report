% Chapter Template

\chapter{Implementation} % Main chapter title

\label{Chapter3} % Change X to a consecutive number; for referencing this chapter elsewhere, use \ref{ChapterX}

In this chapter we describe how we adapted the preexisting motion capture software in order to obtain the desired distorted behaviour.

\section{Distortion model}

As briefly mentioned in chapter \ref{Chapter2}, we are taking advantage of the Egocentric Coordinate formalism in order to introduce our distortion model. Formally, we modify every relative displacement vectors $\vec{v_i}$ such that:

\begin{equation}
\vec{v_i} \vcentcolon= f(\vec{v_i},\gamma )
\end{equation}

\noindent
And $f$ is defined as:
\begin{equation}
f(\vec{v},\gamma ) = \hat{v} * \left||\vec{v}\right|changeme| * 4^\gamma
\end{equation}

\noindent
Where $\vec{v}$ is a vector, $\hat{v}$ is its normalized counterpart, and $\gamma \in \mathbb{R}$. The $4^\gamma$ is here for readability purpose: when $\gamma = 0.5$ the distortion makes each vector
\\\\


\subsection*{Other Functions}

Before resolving to use a simple linear function for our experimentation process, we tried out different functions that we think are of interest for further applications. Two of these functions are descibed here as a reference for further investigation.
\\\\
Description and plots of the Power and Cosine functions.

\section{Reachable Sphere}

Lorem ipsum dolor sit amet, consectetur adipiscing elit. Aliquam ultricies lacinia euismod. Nam tempus risus in dolor rhoncus in interdum enim tincidunt. Donec vel nunc neque. In condimentum ullamcorper quam non consequat. Fusce sagittis tempor feugiat. Fusce magna erat, molestie eu convallis ut, tempus sed arcu. Quisque molestie, ante a tincidunt ullamcorper, sapien enim dignissim lacus, in semper nibh erat lobortis purus. Integer dapibus ligula ac risus convallis pellentesque.
