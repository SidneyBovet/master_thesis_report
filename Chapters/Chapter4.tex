% Chapter Template

\chapter{Experiment} % Main chapter title

\label{Chapter4} % Change X to a consecutive number; for referencing this chapter elsewhere, use \ref{ChapterX}

We are trying to estimate the limits of self attribution of a distorted movement and will do so by estimating the just noticeable difference (JND) using the adaptive staircase method introduced by [ref]. This method changes the intensity of the distortion based on whether the subject judged the last trial as distorted. This last piece of knowledge is gathered using a Yes/No question: "Did the movements you saw exactly correspond to the movements you performed?"
\\\\
While the whole set of IK goals will be distorted during the experiment, we will be focusing on the hand movement.

\section{Hypotheses}

The hypotheses we have for the experiment are the following:

\begin{labeling}{H1}
  \item [H1] The JND will be higher when the distortion is positive.
  \item [H2] The Cosine function will have a higher JND than the Strength one.
\end{labeling}

\section{Experiment design}

We manipulate two factors, namely the orientation of the distortion (positive or negative), and the distortion function itself (Strength or Cosine). See chapter \ref{Chapter3} for a complete overview of these consepts.
\\\\
The experiment itself is perform in a seated position in order to avoid any unnecessary movement of the lower limbs, and has the subject reach six successive target located at various positions both in space and on their skin.
\\\\
The target will alternate between "on skin" (ST) and "in the air" (AT), and the subjects are required to use their right hand to perorm the reaching motion.
\\\\
One experiment run consists of a reaching task where the subjects have to move their right hand from one target to the other. The targets will show in alternating order between ST and AT in order to maximize the amplitude of the movements.

\section{Subjects}

A few physical limitations will be applied to filter the subjects of this experiment. They will be required to be right-handed for ease of software developement, and will need to be both smaller than 180cm and have an body mass index between 18 and 27. The latter is due to our motion capture equipment and especially the suit on which the markers are placed.
\\\\
We also require that they have a normal or corrected to normal vision, and be fluent in both written and spoken english.
