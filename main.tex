%%%%%%%%%%%%%%%%%%%%%%%%%%%%%%%%%%%%%%%%%
% Masters/Doctoral Thesis
% LaTeX Template
% Version 2.3 (25/3/16)
%
% This template has been downloaded from:
% http://www.LaTeXTemplates.com
%
% Version 2.x major modifications by:
% Vel (vel@latextemplates.com)
%
% This template is based on a template by:
% Steve Gunn (http://users.ecs.soton.ac.uk/srg/softwaretools/document/templates/)
% Sunil Patel (http://www.sunilpatel.co.uk/thesis-template/)
%
% Template license:
% CC BY-NC-SA 3.0 (http://creativecommons.org/licenses/by-nc-sa/3.0/)
%
%%%%%%%%%%%%%%%%%%%%%%%%%%%%%%%%%%%%%%%%%

%----------------------------------------------------------------------------------------
%	PACKAGES AND OTHER DOCUMENT CONFIGURATIONS
%----------------------------------------------------------------------------------------

\documentclass[
11pt, % The default document font size, options: 10pt, 11pt, 12pt
%oneside, % Two side (alternating margins) for binding by default, uncomment to switch to one side
chapterinoneline,% Have the chapter title next to the number in one single line
english, % ngerman for German
singlespacing, % Single line spacing, alternatives: onehalfspacing or doublespacing
%draft, % Uncomment to enable draft mode (no pictures, no links, overfull hboxes indicated)
%nolistspacing, % If the document is onehalfspacing or doublespacing, uncomment this to set spacing in lists to single
%liststotoc, % Uncomment to add the list of figures/tables/etc to the table of contents
%toctotoc, % Uncomment to add the main table of contents to the table of contents
parskip, % Uncomment to add space between paragraphs
%nohyperref, % Uncomment to not load the hyperref package
headsepline, % Uncomment to get a line under the header
]{MastersDoctoralThesis} % The class file specifying the document structure

\usepackage[utf8]{inputenc} % Required for inputting international characters
\usepackage[T1]{fontenc} % Output font encoding for international characters

%\usepackage{palatino} % Use the Palatino font by default
\usepackage{fourier}

\usepackage{graphicx}
\usepackage{caption}
\usepackage{subcaption}
\usepackage{wrapfig} % wrapfigure: figures with wrapping text around them
\usepackage{scrextend} % label-based lists
\addtokomafont{labelinglabel}{\sffamily}
\PassOptionsToPackage{hyphens}{url}\usepackage{hyperref} % use hyphens to break url lines
\usepackage{siunitx}
\usepackage{mathtools}
\usepackage{amsmath}
\usepackage{mathtools}
\renewcommand{\vec}[1]{\mathbf{#1}} % in order to have the vectors in bold
\DeclarePairedDelimiter{\norm}{\lVert}{\rVert} % for the nice double bars
\usepackage{array} % for the table in appendix A


\usepackage[backend=bibtex,style=ieee,natbib=true]{biblatex} % Use the bibtex backend with the authoryear citation style (which resembles APA)

\addbibresource{example.bib} % The filename of the bibliography

\usepackage[autostyle=true]{csquotes} % Required to generate language-dependent quotes in the bibliography


%----------------------------------------------------------------------------------------
%	MARGIN SETTINGS
%----------------------------------------------------------------------------------------

\geometry{
	paper=a4paper, % Change to letterpaper for US letter
	inner=2.5cm, % Inner margin
	outer=2.5cm, % Outer margin
	bindingoffset=1.2cm, % Binding offset
	top=1.5cm, % Top margin
	bottom=1.5cm, % Bottom margin
	%showframe,% show how the type block is set on the page
}

%----------------------------------------------------------------------------------------
%	THESIS INFORMATION
%----------------------------------------------------------------------------------------

\thesistitle{Ensuring Self-Haptic Consistency for Immersive Amplified Embodiment} % Your thesis title, this is used in the title and abstract, print it elsewhere with \ttitle
\supervisor{Dr. Ronan \textsc{Boulic}} % Your supervisor's name, this is used in the title page, print it elsewhere with \supname
\examiner{} % Your examiner's name, this is not currently used anywhere in the template, print it elsewhere with \examname
\degree{MSc in Computer Science} % Your degree name, this is used in the title page and abstract, print it elsewhere with \degreename
\author{Sidney \textsc{Bovet}} % Your name, this is used in the title page and abstract, print it elsewhere with \authorname
\addresses{} % Your address, this is not currently used anywhere in the template, print it elsewhere with \addressname

\subject{} % Your subject area, this is not currently used anywhere in the template, print it elsewhere with \subjectname
\keywords{Vitual reality, embodiment, agency, retargeting, motion capture, movement amplification, distortion} % Keywords for your thesis, this is not currently used anywhere in the template, print it elsewhere with \keywordnames
\university{École Polytechnique Fédérale de Lausanne} % Your university's name and URL, this is used in the title page and abstract, print it elsewhere with \univname
\department{\href{http://ic.epfl.ch/en}{School of Computer and Communication Sciences}} % Your department's name and URL, this is used in the title page and abstract, print it elsewhere with \deptname
\group{\href{http://iig.epfl.ch/}{Immersive Interaction Group}} % Your research group's name and URL, this is used in the title page, print it elsewhere with \groupname
\faculty{\href{http://ic.epfl.ch/computer-science}{Computer Science Section}} % Your faculty's name and URL, this is used in the title page and abstract, print it elsewhere with \facname

\hypersetup{pdftitle=\ttitle} % Set the PDF's title to your title
\hypersetup{pdfauthor=\authorname} % Set the PDF's author to your name
\hypersetup{pdfkeywords=\keywordnames} % Set the PDF's keywords to your keywords

\begin{document}

\frontmatter % Use roman page numbering style (i, ii, iii, iv...) for the pre-content pages

\pagestyle{plain} % Default to the plain heading style until the thesis style is called for the body content

%----------------------------------------------------------------------------------------
%	TITLE PAGE
%----------------------------------------------------------------------------------------

\begin{titlepage}
\begin{center}

{\scshape\LARGE \univname\par}\vspace{1.5cm} % University name
\textsc{\Large Master Thesis}\\[0.5cm] % Thesis type

\HRule \\[0.4cm] % Horizontal line
{\huge \bfseries \ttitle\par}\vspace{0.4cm} % Thesis title
\HRule \\[1.5cm] % Horizontal line

\begin{minipage}[t]{0.4\textwidth}
\begin{flushleft} \large
\emph{Author:}\\
{\authorname} % Author name - remove the \href bracket to remove the link
\end{flushleft}
\end{minipage}
\begin{minipage}[t]{0.4\textwidth}
\begin{flushright} \large
\emph{Supervisor:} \\
{\supname} % Supervisor name - remove the \href bracket to remove the link
\end{flushright}
\end{minipage}\\[6cm]
\groupname\\\deptname\\[2cm] % Research group name and department name

{\large Lausanne, EPFL \\[0.3cm] \today}\\[4cm] % Date
%\includegraphics{Logo} % University/department logo - uncomment to place it

\vfill
\end{center}
\end{titlepage}

%----------------------------------------------------------------------------------------
%	DECLARATION PAGE
%----------------------------------------------------------------------------------------

%\begin{declaration}
%\addchaptertocentry{\authorshipname}
%
%\noindent I, \authorname, declare that this thesis titled, \enquote{\ttitle} and the work presented in it are my own. I confirm that:
%
%\begin{itemize}
%\item This work was done wholly or mainly while in candidature for a research degree at this University.
%\item Where any part of this thesis has previously been submitted for a degree or any other qualification at this University or any other institution, this has been clearly stated.
%\item Where I have consulted the published work of others, this is always clearly attributed.
%\item Where I have quoted from the work of others, the source is always given. With the exception of such quotations, this thesis is entirely my own work.
%\item I have acknowledged all main sources of help.
%\item Where the thesis is based on work done by myself jointly with others, I have made clear exactly what was done by others and what I have contributed myself.\\
%\end{itemize}
%
%\noindent Signed:\\
%\rule[0.5em]{25em}{0.5pt} % This prints a line for the signature
%
%\noindent Date:\\
%\rule[0.5em]{25em}{0.5pt} % This prints a line to write the date
%\end{declaration}
%
%\cleardoublepage

%----------------------------------------------------------------------------------------
%	QUOTATION PAGE
%----------------------------------------------------------------------------------------

%\vspace*{0.2\textheight}

%\noindent\enquote{\itshape Thanks to my solid academic training, today I can write hundreds of words on virtually any topic without possessing a shred of information, which is how I got a good job in journalism.}\bigbreak

%\hfill Dave Barry

%----------------------------------------------------------------------------------------
%	ABSTRACT PAGE
%----------------------------------------------------------------------------------------

\begin{abstract}
\addchaptertocentry{\abstractname} % Add the abstract to the table of contents

With the rise of consumer-grade Virtual Reality (VR) technologies, high quality VR equipment now no longer is only accessible to research institutes or universities but to the general public as well. This opens up a lot of possibilities.

One of these opportunities appears to be in the field of stroke rehabilitation. Due to a temporary lack of oxygen in parts of the brain, a stroke may result in partial paralysis of a limb. In such a case the rehabilitation process is long, tedious, and often demotivating. The way we currently treat this condition is through exercises and movements of the affected limb, hence making use of neuroplasticity in order to fully recover the original range of motion. This process however is really demotivating, because the patients are forced to face their handicap. We however believe that there is a simple, yet effective solution that will help stroke patients stay motivated, and that is by providing them with a VR game that transposes and amplifies their movements in a virtual environment.

In the past few years, huge improvements have been achieved in the area of motion capture. New techniques allow for instance to reflect skin-to-skin contacts more precisely than ever, such as the Egocentric Coordinate formalism. We propose a modification of this coordinate system in order to apply a distortion to the movements of a tracked performer, while preserving self-haptic consistency. In other words, we introduce a way to create an application in which, when patients touch their skin the virtual avatar does so as well, but when they lift their hand \SI{10}{\centi\meter} above the surface of the skin, the avatar may show a \SI{20}{\centi\meter} gap if a doubling distortion is applied.

We also propose an experiment in order to understand how well we accept a movement distortion, which is key here because we do not want patients to detect the distortion, or at least not too heavily. Indeed, the goal is to keep them motivated at a reaching task of some sort, but if their eyes constantly remind them that what they see is not what they achieve, we might loose the motivating effect we are looking for.

This work therefore tries to build upon the foundations of the field of amplified embodied interactions, and we hope to see more research in this area, which surely will result in dramatic improvement in terms of treatment time and life quality for stroke patients, as well as for VR experiences in general.

\bigskip
\textbf{Keywords}: \keywordnames

\end{abstract}

%----------------------------------------------------------------------------------------
%	ACKNOWLEDGEMENTS
%----------------------------------------------------------------------------------------

\begin{acknowledgements}
\addchaptertocentry{\acknowledgementname} % Add the acknowledgements to the table of contents

I would like first of all to thank Dr.\ Ronan Boulic for giving me the opportunity to work on this project, not only as a Master student at the time of writing this, but also as a Research Assistant in a few months. I further thank him for his supervision, support, and guidance during this project. My thanks also go to Prof.\ St{\'e}phane Gobron, who has accepted to assist Dr.\ Boulic in reviewing and grading the present report. I am proud to defend my Master Thesis in front of these people.

I also wish to thank Eray Molla and Henrique Debarba, who have provided valuable and continuous support in their respective areas of expertise as former lab members, and I wish good luck to Thibault for his future research work.

On a more personal note, I wish to express my gratitude to Alexandre, John, and Thomas for the countless hours of philosophical discussions about theoretical computer science and physics, and additionally to Florian, Fabien and Ga{\"e}l for the numerous sessions of frenzy dice-rolling. Likewise, I take the opportunity to thank Mathieu, Zhivka, and Joanna for the various project I had the chance to work on alongside these talented computer scientists. My thanks also go to Joss, Ogier, Yann, and Maxime for their moral support and enjoyable company---at Sat among other places.

To conclude I would like to express my deepest gratitude to Thierry, Dora, Johanna, and No{\'e}mie for their flawless support. I am greatly indebted to these people.

\end{acknowledgements}

%----------------------------------------------------------------------------------------
%	LIST OF CONTENTS/FIGURES/TABLES PAGES
%----------------------------------------------------------------------------------------

\tableofcontents % Prints the main table of contents

%\listoffigures % Prints the list of figures

%\listoftables % Prints the list of tables

%----------------------------------------------------------------------------------------
%	ABBREVIATIONS
%----------------------------------------------------------------------------------------

%\begin{abbreviations}{ll} % Include a list of abbreviations (a table of two columns)
%
%\textbf{LAH} & \textbf{L}ist \textbf{A}bbreviations \textbf{H}ere\\
%\textbf{WSF} & \textbf{W}hat (it) \textbf{S}tands \textbf{F}or\\
%
%\end{abbreviations}

%----------------------------------------------------------------------------------------
%	PHYSICAL CONSTANTS/OTHER DEFINITIONS
%----------------------------------------------------------------------------------------

%\begin{constants}{lr@{${}={}$}l} % The list of physical constants is a three column table
%
%% The \SI{}{} command is provided by the siunitx package, see its documentation for instructions on how to use it
%
%	Speed of Light & $c_{0}$ & \SI{2.99792458e8}{\meter\per\second} (exact)\\
%%Constant Name & $Symbol$ & $Constant Value$ with units\\
%
%\end{constants}

%----------------------------------------------------------------------------------------
%	SYMBOLS
%----------------------------------------------------------------------------------------

%\begin{symbols}{lll} % Include a list of Symbols (a three column table)
%
%$a$ & distance & \si{\meter} \\
%$P$ & power & \si{\watt} (\si{\joule\per\second}) \\
%%Symbol & Name & Unit \\
%
%\addlinespace % Gap to separate the Roman symbols from the Greek
%
%$\omega$ & angular frequency & \si{\radian} \\
%
%\end{symbols}

%----------------------------------------------------------------------------------------
%	DEDICATION
%----------------------------------------------------------------------------------------

%\dedicatory{For/Dedicated to/To my\ldots}

%----------------------------------------------------------------------------------------
%	THESIS CONTENT - CHAPTERS
%----------------------------------------------------------------------------------------

\mainmatter % Begin numeric (1,2,3...) page numbering

\pagestyle{thesis} % Return the page headers back to the "thesis" style

% Include the chapters of the thesis as separate files from the Chapters folder
% Uncomment the lines as you write the chapters

% Chapter 1

\chapter{Introduction} % Main chapter title

\label{Chapter1} % For referencing the chapter elsewhere, use \ref{Chapter1}

%----------------------------------------------------------------------------------------
%	SECTION 1
%----------------------------------------------------------------------------------------

Foreword: will cover multiple subjects from stroke rehabilitation to motion capture.

\section{Stroke Rehabilitation}

Explanation of what a stroke is and how to recover.

\subsection{Motivation}

Emphasis on the motivation being a key aspect of the rehabilitation process,
with [ref].

\section{Motion Caputre}

The action of capturing one's movements and possibly altering it, be it in order to fit
it onto an avatar of different moprphology or to change aid one reaching a target.

\section{Psychometrics}

JND and other PM concepts.

% Chapter Template

\chapter{Egocentric Coordinates}

\label{Chapter2}

In this chapter we briefly cover the Inverse Kinematics problem and how its parameters are acquired, and we then dive more deeply in the description of a special coordinate system based on the relative positions of different body parts around a point.

\section{Motion Capture and Inverse Kinematics}

As stated by Paul \cite{paul1981robot}, Inverse Kinematics (IK) refers to the use of kinematic equations of a chain of rigid bodies in order to produce the desired pose of that chain so that it satisfies a position---or goal---for that chain's end effector. Such techniques are useful for instance to compute the joint position of an industrial robot whose task is to tighten a screw at a given location. Another, more interesting application to us of such algorithms is to compute the pose of a virtual body. An avatar is described as a kinematic chain, and its feet, elbows, knees or head positions are considered as IK goals. An IK solver may be numeric \cite{goldenberg1985complete,manocha1994efficient}, or it may be analytical, such as the one proposed by Molla et al.\ \cite{molla2013singularity}. In this case, a closed-form expression is provided that takes one or several IK goals as input, and outputs a vector of joint positions and orientations to be set in order to satisfy the goals.

These desired positions may be defined by some game engine so that a character points a tool at a desired location, but they are also often obtained using motion capture techniques, as mentioned in Section \ref{sec:mocap}. Such techniques usually take advantage of special suits equipped with markers, whose individual positions are acquired either by the marker itself, or by using tracking devices. As explained by \cite{west1995motion}, a suit may for instance be fitted with magnetic sensors and used in a room where a coil emits a given magnetic field, so that each sensor knows its position and orientation relative to the emitting base, acting as an absolute reference. Another way to acquire a position is to place an infrared LED at that location and tracking it using several cameras. If the position of each tracker is known a priori (e.g.\ through space calibration) then the marker's position can be recovered using triangulation.

All the above techniques focus on getting the absolute position (and sometimes orientation) of an IK goal. This is a necessary step for any animation application, but an end effector's position need not be expressed as an absolute value throughout the application's pipeline. The semantic information conveyed by a performer may in fact be better preserved by using an alternative, relative coordinate system.

\section{Egocentric Coordinates}
\label{sec:egocentric}

The idea of using reference points others than the origin of the world's axes in order to describe an IK goal is already a few years old \cite{al2013relationship}, but Molla et al.\ \cite{molla2017egocentric,molla2016precise} took it a step further by using one's own body parts as reference points. This allows to correctly map the semantics of performed motions onto avatars of different proportions and sizes. Such body-centered coordinate system is called Egocentric.

A crude body representation is computed from markers placed on the performer. The limbs are represented as capsules while the trunk and head are sampled at multiple locations and then translated into a series of triangles forming a crude mesh. Given such body representation with $n$ body parts, the position $\vec{p}_j$ of an IK goal $j$ is then represented as shown in Equation \ref{eq:EgocentricPosition}.

\begin{equation}
\label{eq:EgocentricPosition}
\vec{p}_j = \displaystyle\sum_{i=1}^{n} \hat{\lambda}(\vec{x}_i + \vec{v}_i)
\end{equation}

The closest point on surface $i$ is denoted $\vec{x}_i$, while $\vec{v}_i$ is the relative displacement vector going from $\vec{x}_i$ to $\vec{p}_i$, and $\lambda $ is a normalization factor. This last value is computed as $\lambda = \lambda_p \cdot \lambda_\perp$, whose proximity and orthogonality factors are defined in Equations \ref{eq:lambda_p} and \ref{eq:lambda_t}. $\hat{\lambda}$ is obtained by further linearly normalizing $\lambda$ such that $\sum \hat{\lambda} = 1$.

\begin{align}
    \label{eq:lambda_p}
    \textbf{Proximity: } \lambda_p &=
    \begin{cases}
        \frac{1}{\epsilon}          &\text{if } \norm{\vec{v}} \leq \epsilon\\
        \frac{1}{\norm{\vec{v}}}    &\text{otherwise}
    \end{cases}
    \\
    \label{eq:lambda_t}
    \textbf{Orthogonality: } \lambda_\perp &=
    \begin{cases}
        \cos(\epsilon)      &\text{if } \cos(\alpha) \leq \epsilon\\
        \cos(\alpha)        &\text{otherwise}
    \end{cases}
\end{align}

In Equation \ref{eq:lambda_t}, $\alpha$ denotes the angle between the surface normal $\vec{n}$ at the closes point $\vec{x}$ of a body part, and the relative displacement vector $\vec{v}$. The presence of $\epsilon$, a tiny constant, helps prevent instability due to floating point precision numbers. The justification for measuring orthogonality is that (1) if a surface normal points at a joint, chances are they are interacting in some way, and (2) orthogonality holds semantic information about gestures, such as holding a hand in front of the heart.

The movement deformation we introduce in this work relies heavily on this Egocentric representation of the IK goals, and more precisely acts on the relative displacement vectors $\vec{v}$ described above. The next chapter describes our distortion model, as well as exactly how we adapted the Egocentric formalism in order to apply this distortion to an avatar's movements.
% Chapter Template

\chapter{Implementation} % Main chapter title

\label{Chapter3} % Change X to a consecutive number; for referencing this chapter elsewhere, use \ref{ChapterX}

In this chapter we begin by describing our distortion model and what function we used, and then we show how we adapted the preexisting motion capture software in order to obtain the desired distorted behavior.

\section{Distortion model}


As briefly mentioned in chapter \ref{Chapter2}, we are taking advantage of the Egocentric Coordinate formalism in order to introduce our distortion model. We modify each relative displacement vectors $\vec{v}_i$ according to some value $\gamma$. A distorted position $\vec{p}_j$ is thus obtained using equation \ref{eq:DistortionOperation}, which has been obtained by modifying Equation \ref{eq:EgocentricPosition} using a function that we are going to detail in the next few lines.

\begin{equation}
\label{eq:DistortionOperation}
\vec{p}_j = \displaystyle\sum_{i=1}^{n} \hat{\lambda}\big(\vec{x}_i + D(\vec{v}_i,\gamma )\big)
\end{equation}

Figure \ref{fig:armExamples} shows an example of a distorted position obtained by changing the length of all vectors $\vec{v}_i$.

\begin{figure}[h]
    \centering
    \begin{subfigure}[b]{0.2\textwidth}
        \includegraphics[width=\textwidth]{Figures/simple_distortion_3.png}
        \caption{$\gamma = 3$}
    \end{subfigure}
    ~ %add desired spacing between images, e. g. ~, \quad, \qquad, \hfill etc.
    %(or a blank line to force the subfigure onto a new line)
    \begin{subfigure}[b]{0.2\textwidth}
        \includegraphics[width=\textwidth]{Figures/simple_distortion_-3.png}
        \caption{$\gamma = -3$}
    \end{subfigure}
    \caption{Two examples of our linear distortion applied to a simple IK arm with multiple segments. The gray lines are the relative displacement vectors and the red ones are their distorted counterparts. Similarly, the gray arms represent the pose the real arm would take whereas the red ones show the two resulting distorted poses.}\label{fig:armExamples}
\end{figure}

We will now describe the function we will be using in our subsequent experiment, and then propose a distortion expression that may be better-suited for real-world applications.

\subsection{Linear Function}

For ease of experimentation and understandability, we are looking for a linear function $f(x) = ax + b$. Figure \ref{fig:armExamples} gives an example of what we aim to achieve, while Figure \ref{fig:plotsOfGamma} below gives a more mathematical point of view of the distortion we are looking for, especially in terms of $a$, the slope of the function. This plot, as well as all of the other plots of this report, were obtained using the Plotly API \cite{plotly}.

\begin{figure}[h]
    \center{\includegraphics[width=.7\textwidth]
    {Figures/gamma_values.png}}
    \caption{An example of a few distortion functions for various values of slope $a$ and gain $\gamma $.}\label{fig:plotsOfGamma}
\end{figure}

First of all we want to preserve self-haptic contacts. Such contact happens when a relative displacement vector satisfies $\norm{\vec{v}} = 0$, which means that we need $f(0) = 0$, and thus $b = 0$.

Intuitively, the slopes should be arranged around $a=1$, which results in no distortion at all and that we want to correspond to $\gamma = 0$. One can also figure out that there is a correspondence between slopes above and below the line $f(x) = x$. For instance, for a given virtual distance to cover, a slope of \num{0.5} makes the traveling distance twice as long, whereas a slope of \num{2} halves the required movement.

Formally, we are modifying each relative displacement vector as specified in Equation \ref{eq:DistortionOperation}, with $\gamma$ representing a gain, measured in \SI{}{\decibel}, and $f$ defined by Equation \ref{eq:DistortionFunction} below.

\begin{equation}
\label{eq:DistortionFunction}
D(\vec{v},\gamma ) = \hat{\vec{v}} \cdot \norm{\vec{v}} \cdot 10^{\frac{\gamma}{10}}
\end{equation}

In this equation, $\vec{v}$ is a vector, $\hat{\vec{v}}$ is its normalized counterpart, and $\gamma \in \mathbb{R}$. The last factor, $10^{\frac{\gamma}{10}}$, comes from the definition of a gain, in \SI{}{\decibel} \cite{book:decibel}, based on two values $P_1$ and $P_2$ of a single, yet undefined unit:

\begin{align*}
    \text{gain} = \gamma &= 10 \cdot \log_{10} (\frac{P_1}{P_2})\\
    \frac{\gamma}{10} &= \log_{10} (\text{slope})\\
    \text{slope} &= 10^{\frac{\gamma}{10}}.
\end{align*}

A value of $\gamma = 3$ thus indicates that the virtual movement will roughly be twice the amplitude of the registered one ($1.995 \approx 2$), while a gain of $\gamma = -3$ means one will have to travel twice as big a distance (\num{0.501}) as perceived in order to cover it. Figure \ref{fig:armExamples} shows two examples of distortion, and Figure \ref{fig:plotsOfGamma} gives a few instances of this linear distortion functions with varying values of $\gamma $ and the corresponding slope $a$.

\subsection{Other Functions}
\label{sec:otherFunctions}
Before deciding to use a simpler, thus easier to quantify, linear function for our experimentation process, we tried out different functions that we think are of interest for further applications. Two of these functions are described here as a reference for further investigation.

As for our linear function, we require the distortion to be null around $\norm{\vec{v}} = 0$. Similarly, we introduce an action range $a_r$ after which the distortion should be null again. The general form of the function applied to our relative displacement vectors then becomes the one described in Equation \ref{eq:distortionCase}.

\begin{equation}
    \label{eq:distortionCase}
    D(\vec{v}, s, a_r) =
    \begin{cases}
        \hat{v} \cdot \norm{\vec{v}} \cdot f(\norm{\vec{v}}, s, a_r)    &\text{if } \norm{\vec{v}} \leq a_r\\
        \vec{v}                                                         &\text{otherwise}
    \end{cases}
\end{equation}

Note that we changed the `$\gamma $' parameter for `$s$', which is due to this parameter no longer denoting a cleanly defined gain, but a vaguer concept of strength. Two notable instances of the function $f$ were implemented. They are shown in Figure \ref{fig:otherDistortions} and are described hereafter.

\begin{figure}[h]
    \centering
    \begin{subfigure}[b]{.45\textwidth}
        \includegraphics[width=\textwidth]{Figures/exponential_distortion.png}
        \caption{Exponential}
        \label{fig:otherDistortionsExp}
    \end{subfigure}
    ~
    \begin{subfigure}[b]{.45\textwidth}
        \includegraphics[width=\textwidth]{Figures/cosine_distortion.png}
        \caption{Cosine}
        \label{fig:otherDistortionsCos}
    \end{subfigure}
    \caption{Multiple instances of the two alternative distortion functions we propose in section \ref{sec:otherFunctions}. Both have been plotted using $a_r = 1$, only varying the strength parameter $s$.}
    \label{fig:otherDistortions}
\end{figure}

\subsubsection{Exponential}

Based on an expression proposed by Khoury \cite{khoury2015human}, this function has the following form:

\begin{equation*}
    f(d, a_r, s) = a_r \cdot \Bigg(\frac{d}{a_r}\Bigg)^{2^{-s}}
\end{equation*}

\subsubsection{Cosine}

Observations made on the previous function lead us to designing this second function.

Obviously, being smoother in terms of position near both ends of the distorted area leads to trade-offs. In our case the function introduces higher velocity discrepancies 

\section{Egocentric Normalization Factor}

We added one more modification to the definition of the position proposed by \cite{molla2017egocentric} which we modified to obtain Equation \ref{eq:DistortionOperation}, and more precisely the way $\lambda $ is defined. As originally explained by \cite{molla2016precise} as well as in Section \ref{sec:egocentric}, it is computed as the product of two importance factors, proximity and orthogonality, respectively denoted $\lambda_p$ and $\lambda_\perp $.

The former importance factor was initially defined as $\lambda_p = \frac{1}{\norm{\vec{v}}}$. In practice we find that this formula does not give enough importance to nearby body parts, and we decided to change it slightly as $\lambda_p = \frac{1}{\norm{\vec{v}}^2}$. --Mention angle solide.-- % which more closely represents the amount of surface of an object that is visible at a distance $\norm{\vec{v}}$.

\section{Reachable Sphere}

A few words on the concept of reachable sphere and how it might help at the limits of the reachable space.

% Chapter Template

\chapter{Experiment} % Main chapter title

\label{Chapter4} % Change X to a consecutive number; for referencing this chapter elsewhere, use \ref{ChapterX}

We are trying to estimate the limits of self attribution of a distorted movement and will do so by estimating the just noticeable difference (JND) in visual stimuli discrepency. This means estimating the just noticeable distortion made to the movement and hence the visual stimuli. The JND is estimated by using the adaptive staircase method introduced by [ref].
\\\\
This method tries to estimate the JND by finding an upper and a lower bound for that value. These are found by changing the intensity of the distortion, based on whether the subject judged the last trial as distorted or not, and the JND is computed as the mean of the last few staircase turns (i.e.\ going from an increasing trend to a decreasing one or vice-versa). The dection judgment is gathered using a Yes/No prompt called the detection question : "Did the movements you saw exactly correspond to the movements you performed?"

\section{Just Noticeable Difference}

The JND will be measured in term of $\gamma$, the gain of the distortion function, Equation \ref{eq:DistortionFunction}. In general, if $\gamma = -3$ the subjects are hindered by having to travel two times the distance between the targets, whereas if $\gamma = 3$ the movement will be amplified and the required motion will be reduced by $50\%$.
\\\\
Due to the nature of the Egocentric Coordinates and how the distortion is applied (respectively detailed in Chapters \ref{Chapter2} and \ref{Chapter3}), this will not exactly be a metric of the difference in the distance that the subjects have to cover in order to reach the target, such as the metric used by [ref to Henrique]. It however gives a good understanding of the strength and the effect of the applied distortion.

\section{Hypothesis}

The hypothesis we have for the experiment is the following:

\begin{labeling}{H1}
  \item [H1] The absolute value of the JND will be higher when the distortion is positive.
\end{labeling}

\section{Equipment and Software}

The HMD used for this experiment is the Oculus Rift in its first consumer version, with a resolution of 0 x 0 pixels per eye and a refresh rate of \SI{90}{\hertz}.
Some more words on head and body tracking, with references to Eray's paper [ref].

\section{Experiment design}

We manipulate two factors: the sign of the distortion (positive or negative), respectively yielding a helped and hindered movement, and the starting position of the task (chest of leg). See chapter \ref{Chapter3} for a complete overview of the concept of distortion and its sign.
\\\\
We consider the starting position as a factor because of the nature of the distortion, which causes horizontal movements to be slightly more distorted. This is due to the proximity of many chest-located reference points as opposed to only four leg segments, thus having the sum of all relative displacement vectors diverting the hand position in the forward direction.

\subsection{Task}

While the whole set of IK goals will be distorted during the experiment, we will be focusing on the dominant hand movement. The task is performed in a seated position in order to avoid any unnecessary movement of the lower limbs, and has the subjects reach three successive target, one of which is in the air in front of them, and the two others are located at various locations on their skin. The reaching task is performed with the directing hand, and the subjects are instructed to keep their other hand at their side.
\\\\
The first target T1 to be displayed is one on the skin and requires the subjects to perform a self-contact in order to activate it. After a random time between \num{200} and \SI{300}{\milli\second}, the target activates. The position of the air target T2 is then computed such that the subjects have to move a predefined distance $d = \SI{50}{\centi\meter}$ between T1 and T2, as well as between T2 and T3 once the air target is reached. This is achieved by computing the intersection between the two spheres of radius $d$ centered on T1 and T3, and chosing the topmost position. A more detailed explanation of how this is achieved can be found in chapter \ref{Chapter3}.
\\\\
One experiment run consists of a reaching task, followed by the detection question. Based on the answer to this question, the experiment software modifies the distortion as follows:

\begin{labeling}{"No" and $\gamma = 0$}
  \item ["Yes"] The discrepancy is increased.
  \item ["No" and $\gamma \neq 0$] The discrepancy is decreased.
  \item ["No" and $\gamma = 0$] The parameter is not changed, given that this would invert the sign of the distortion.
\end{labeling}
\noindent
The amount of each increment or decrement is dynamic: it starts at \num{0.1} and is halved after the first staircase turn. That value is then kept for the rest of that staircase. The staicase is completed either when the subjects change direction 7 times or when they performed 20 trials in that staircase.

\subsection{Procedure}

The subjects are welcomed and introduced to the protocol described here, and then introduced to the tracking equipment. A characterization form is then filled in by the subjects during this first part of the experiment, with background questions regarding any previous VR experiment or experience with HMDs. They are then asked to remove their shoes and to wear the motion capture suit. A calibration is then performed as described by [ref to Eray's paper].
\\\\
Before beginning the actual staircase trials, a familiarization phase takes place. The subjects briefly interact with the virtual environment without any distortion, so that they correctly embody the avatar and understand the question process.
\\\\
More on the next steps later.

\section{Subjects}

A few physical limitations will be applied to filter the subjects of this experiment. They will be required to be right-handed for ease of software developement, and will need to be both smaller than 180cm and have an body mass index between 18 and 27. The latter is due to our motion capture equipment and especially the suit on which the markers are placed.
\\\\
We also require that they have a normal or corrected to normal vision, and be fluent in both written and spoken english.

% Chapter Template

\chapter{Results and Discussion}
\label{Chapter5}

As briefly mentioned towards the end of Section \refsec:reachableSphere}, the code base used in this project, and particularly the retargeting part implementing the whole Egocentric Coordinate formalism, is problematic. Although well formated and making use of carefully chosen variable names, the code is intrinsically complex and has sadly not been well documented nor commented.

This unfortunately forced us to reconsider the schedule for this project, and prevented us from performing the experiment described in Chapter \ref{Chapter4}. Luckily, we will be able to continue this project in the next six months thanks to a project grant by the Hassler Foundation, hence making it possible to run the experiment and publish its results in a subsequent report.

The remainder of this chapter therefore does not describe the results of the experiment itself, but what we have been able to achieve software-wise in terms of distortion and experiment implementation.

\section{Distortion}

A few examples of distortion and the resulting poses have already been shown on Figures \ref{fig:armExamples} and \ref{fig:divergence}. We propose here a more detailed summary of our results and a discussion of them.


\begin{figure}[h]
    \centering
    \begin{subfigure}[b]{.3\textwidth}
        \includegraphics[width=\textwidth]{Figures/distortions/distortions-6.png}
        \caption{$\gamma = -6$}
    \end{subfigure}
    ~
    \begin{subfigure}[b]{.3\textwidth}
        \includegraphics[width=\textwidth]{Figures/distortions/distortions-3.png}
        \caption{$\gamma = -3$}
    \end{subfigure}
    ~
    \begin{subfigure}[b]{.3\textwidth}
        \includegraphics[width=\textwidth]{Figures/distortions/distortions-1.png}
        \caption{$\gamma = -1$}
    \end{subfigure}
    
    
    \begin{subfigure}[b]{.3\textwidth}
        \includegraphics[width=\textwidth]{Figures/distortions/distortions1.png}
        \caption{$\gamma = 1$}
    \end{subfigure}
    ~
    \begin{subfigure}[b]{.3\textwidth}
        \includegraphics[width=\textwidth]{Figures/distortions/distortions3.png}
        \caption{$\gamma = 3$}
    \end{subfigure}
    ~
    \begin{subfigure}[b]{.3\textwidth}
        \includegraphics[width=\textwidth]{Figures/distortions/distortions6.png}
        \caption{$\gamma = 6$}
        \label{subfig:gamma6}
    \end{subfigure}
        
    \caption{An example of distorted poses. As always, the captured arms are in gray, the distorted ones in red, and the red and gray lines respectively represent the distorted and original Egocentric coordinates.}
    \label{fig:distortionExamples}
\end{figure}

As one can observe on Figure \ref{fig:distortionExamples}, the distortion works the way it is expected to. As a reminder, a gain of $\pm\SI{6}{\decibel}$ corresponds to a movement whose amplitude is respectively multiplied by $3.981$ and $0.251$, which means it is changed almost fourfold in each direction with respect to a non distorted one. A value of $\gamma = \pm\SI{1}{\decibel}$ similarly modifies a movement by $1.259$ or $0.794$.

The behavior of the arm in Figure \ref{subfig:gamma6} is precisely the one we proposed to avoid using the reachable sphere described in Section \ref{sec:reachableSphere} but we do not report on it here because, as previously justified, we did not implement it in our experiment.

An interactive web application where one may play with both the real arm position and the gain of the distortion, as well as toggling the presence or not of the reachable sphere, is available at \url{https://sidneybovet.github.io/amplified-embodiment/}. The reader is encouraged to play with the IK chain in order to get a better feeling for how the distortion works, especially how it behaves near the skin.

In order to better understand how such a distortion applies to a captured motion, Figure \ref{fig:realMocapDistortion} shows a subject performing a reaching motion towards the air target. The virtual hand is at the same position on all three pictures, but as one may observe, the subject has his hand at different positions. This is due to the fact that in one case (higher hand position) no distortion was applied, while the other (lower hand position) sees a distortion of \SI{3}{\decibel}. The central image shows the superimposition of the two poses in order to better see the position discrepancy.

\begin{figure}[h]
    \centering
    \begin{subfigure}[b]{.3\textwidth}
        \includegraphics[width=\textwidth]{Figures/handPositionTargetNoDist.png}
    \end{subfigure}
    ~
    \begin{subfigure}[b]{.3\textwidth}
        \includegraphics[width=\textwidth]{Figures/handPositionTwice.png}
    \end{subfigure}
    ~
    \begin{subfigure}[b]{.3\textwidth}
        \includegraphics[width=\textwidth]{Figures/handPositionTarget.png}
    \end{subfigure}
        
    \caption{Pictures of a subject, taken from the same point of view, reaching for the same target but with two different distortions of \SI{0}{\decibel} and \SI{3}{\decibel}, with a visual representation of where the target is (yellow dot). The left image shows the \SI{0}{\decibel}, the right one has a distortion of \SI{3}{\decibel}, and the center one is the superimposition of both images.}
    \label{fig:realMocapDistortion}
\end{figure}

This means that in order to perform the same perceived movement, the subject once had to cover the whole distance between his leg and the target, and could in a second time roughly travel two times less in order to reach that target.

\section{Experiment}

---link to the summary video of the experiment---

% Chapter Template

\chapter{Conclusion} % Main chapter title

\label{Chapter6}

We believe the content of this report is promising. The results obtained in terms of distortion behavior are satisfying and preliminary testing was very interesting and corroborates the findings of Debarba \cite{debarba2017embodiment}, stating that a helping distortion is more easily accepted than a hindering one.

While additional work is required before a complete understanding of this phenomenon occurs, we are confident that the second part of this project will lead to interesting findings and will eventually allow the rise of a new type of therapy. As explained in Chapter \ref{Chapter1}, there is room for improvement in the field of stroke rehabilitation.

We could for instance imagine a simplified calibration process involving only the upper limbs, along with a cheaper tracking solution such as the ones proposed by consumer-grade VR products, that would allow patients to train various movements at home. This dramatically reduces the cost of such therapy while allowing the clinicians to have remote access to detailed reports on their patient's activity, with metrics like movement speed or distance covered with the affected limb. Such metrics may also be key in keeping the patients motivated, in that they can track the said measurements and see how they are improving.

\section{Further Work}

Before being able to issue a rehabilitation VR application we must better understand to what extent the human perception allows such distortions to happen. This is going to be achieved through performing the experiment described in Chapter \ref{Chapter4} as a pilot study, and then adapting the protocol to the results we will find. Points that may need refinement include the method and metric used (adaptive staircase and gain), the different target placements, and other experiment details such as the question asked.

A prerequisite to running the experiment however is to solve the issues exposed in section \ref{sec:issues}. Priority will be given to fixing these.

In parallel, a student's project will be focusing on the calibration process, which we hope to make faster and more accurate. If the project is successful, it will be included in our procedure.

On a wider scope, additional investigation will be required in order to fully understand the impact of this and other distortions on the Sense of Embodiment, and more precisely on its Agency component. More complex interactive experimentation should be conducted, and it might be interesting to assess the SoE trough physiological measurements (e.g.\ galvanic skin response) while manipulating the strength of the distortion. Such measurement have already been carried out by Debarba et al. \cite{debarba2017perspective} in the context of perspective changes, and were found useful in assessing the differences between the conditions.

%----------------------------------------------------------------------------------------
%	THESIS CONTENT - APPENDICES
%----------------------------------------------------------------------------------------

\appendix % Cue to tell LaTeX that the following "chapters" are Appendices

% Include the appendices of the thesis as separate files from the Appendices folder
% Uncomment the lines as you write the Appendices

% Appendix A

\chapter{Questionnaires} % Main appendix title

\label{AppendixA} % For referencing this appendix elsewhere, use \ref{AppendixA}

\section{Characterization Questionnaire}

\begin{table}[h]
  \centering
  \caption{The characterization questionnaire used at the begining of the experiment.}
  \begin{tabular}{| p{.4\textwidth} | m{.6\textwidth} |}
    \hline
    \textbf{Question}                                 & \textbf{Answer} \\ \hline
    Identifier                                        & Numeric answer \\ \hline
    Age                                               & Numeric answer \\ \hline
    Gender                                            & Male / Female \\ \hline
    How often do you participate on experiments
    using Virtual Reality equipments?                 & Never / A few times / Every month / week / day \\ \hline
    How often do you use head mounted displays?       & Never / A few times / Every month / week / day \\ \hline
    How often do you play video games?                & Never / A few times / Every month / week / day \\ \hline
    How often do use the Microsoft Kinect, Nintendo
    Wii or Playstation move?                          & Never / A few times / Every month / week / day \\ \hline
    Hand of preference?                               & Left / Right \\ \hline
    Area(s) of expertise / study / work / interest?   & Text answer \\ \hline
    Are you a student?                                & Bachelor / Master / PhD / No \\ \hline
  \end{tabular}
\end{table}

\section{Detection Question}

The detection question introduced in Chapter \ref{Chapter4} that was asked multiple times to the subjects was the following : ``Did the movements you saw exactly correspond to the movements you performed?''

%% Appendix Template

\chapter{Results} % Main appendix title

\label{AppendixB} % Change X to a consecutive letter; for referencing this appendix elsewhere, use \ref{AppendixX}

Results will be fully shown here.


%----------------------------------------------------------------------------------------
%	BIBLIOGRAPHY
%----------------------------------------------------------------------------------------

\printbibliography[heading=bibintoc]

%----------------------------------------------------------------------------------------

\end{document}
